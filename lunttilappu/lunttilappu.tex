\documentclass{article}
\usepackage[utf8]{inputenc}
\usepackage[T1]{fontenc}
\usepackage[finnish]{babel}
\usepackage[a4paper,landscape,margin=1cm]{geometry}
\usepackage{multicol}

% tuhoaa ylä- ja alatunnisteet
\usepackage{fancyhdr}
\pagestyle{fancy}
\fancyhead{}
\fancyfoot{}
\renewcommand{\headrulewidth}{0pt}

% parit komennot
\newcommand{\com}[1]{{\large{\texttt{#1}}}}
\newcommand{\Com}[1]{\com{#1}  }
\newcommand{\M}{{\large{\textit{M}}}}

\begin{document}
%\section*{Vi ja Vim -- yleisimmät komennot}

\begin{multicols}{4}

Muistakaa komentojen yleissyntaksi: \com{\#k\M} tai \com{k\#\M}, missä \com{k} komento, \com{\M} liike ja \com{\#} lukumäärä eli liikkeen kerroin.

\subsection*{Liikkuminen (\M)}
Normaalimoodissa.
\begin{description}
\item{\Com{j k}}  alas / ylös 
\item{\Com{h l}}  vasemmalle / oikealle 
\item{\Com{w}}  seuraavan sanan alkuun 
\item{\Com{b}}  nykyisen sanan alkuun 
\item{\Com{e}}  nykyisen sanan loppuun 
\item{\Com{W}}  seuraavan ison sanan alkuun 
\item{\Com{B}}  nykyisen ison sanan alkuun 
\item{\Com{E}}  nykyisen ison sanan loppuun 
\item{\Com{( )}}  virkkeen alkuun / loppuun
\item{\Com{\{ \}}}  kappaleen alkuun / loppuun
\item{\Com{0 \$}} Rivin alkuun / loppuun
\item{\Com{gg}}  Tiedoston alkuun 
\item{\Com{G}}  Tiedoston loppuun 
\item{\Com{\#gg}} Riville \#
\item{\Com{f\emph{a}}} Seuraavaan a-kirjaimeen
\item{\Com{F\emph{a}}} Edelliseen a-kirjaimeen
\item{\Com{H}} Ruudun yläreunaan
\item{\Com{M}} Ruudun keskelle
\item{\Com{L}} Ruudun alareunaan
\end{description}

\subsection*{Lisäys}
Lisäysmoodista poistutaan painamalla Esc-näppäintä.
\begin{description}
\item{\Com{i}}  aloita tähän kohtaan lisäys 
\item{\Com{a}}  aloita seuraavasta kohdasta lisäys 
\item{\Com{I}}  lisää rivin alkuun 
\item{\Com{A}}  lisää rivin loppuun 
\item{\Com{o}}  uusi rivi nykyisen rivin alle 
\item{\Com{O}}  uusi rivi nykyisen rivin ylle 
\end{description}

\subsection*{Muutokset}
Oliot \M: joko liikekomentoja tai tekstiobjekteja. Kerroin joko tämän eteen tai koko komennon eteen.
\begin{description}
\item{\Com{r\emph{x}}} Korvaa merkki $x$:ksi
\item{\Com{x}} Poista merkki
\item{\Com{cc}}  Korvaa rivi 
\item{\Com{c\M}}  Korvaa olio 
\item{\Com{C}} Korvaa loppurivi
\item{\Com{dd}}  Poista rivi 
\item{\Com{d\M}}  Poista olio 
\item{\Com{D}} Poista loppurivi
\end{description}

\subsection*{Rekisterit}
Rekisterit määrätään lisäämällä ennen komentoa lainausmerkki ja  rekisterin nimi: \com{\textquotedbl a} olisi rekisteri a. Iso aakkonen täydentää entisen perään, pieni aakkonen korvaa sisällön. Ilman tätä määrettä mennään oletusrekisterillä \com{\textquotedbl 0}
\begin{description}
\item{\Com{y\M}} Kopioi (yank) olio
\item{\Com{yy}} Kopioi rivi
\item{\Com{d\M}} Leikkaa olio
\item{\Com{dd}} Leikkaa rivi
\item{\Com{p}} Liitä jälkeen
\item{\Com{P}} Liitä tähän
\item{\Com{:reg}} Lista rekistereistä
\end{description}

\subsection*{Tekstiobjektit (\M)}
Syntaksi: \com{a}x tai \com{i}x, a niin kuin \emph{a, englannin kielen artikkeli} ja i niin kuin \emph{inner, sisukset}. \emph{A paragraph}: \com{ap}. A ottaa siis vähän tyhjää väliä jälkeen mukaansa, I ei niinkään.

\begin{description}
\item{\Com{w/W}}  sana / isosana
\item{\Com{p}}  kappale 
\item{\Com{s}}  virke 
\item{\Com{(}}  sulut  (ja vastaavasti \com{) \{ \}})
\item{\Com{\textquotedbl}}  lainaus
\end{description}

\subsection*{Visual-moodi}
Valitse ensin haluamasi alue liikkeillä ja tekstiobjekteilla, sitten haluamasi komento.
\begin{description}
\item{\Com{v}} Asetu visual-moodiin
\item{\Com{V}} Asetu visual-moodiin, jossa valitaan vain rivejä
\item{\Com{C-v}} Lohkomoodi, valitse mielivaltainen nelikulmio
\end{description}

\subsection*{Etsintä ja korvaus}

\begin{description}
\item{\Com{/\emph{haku}}} Hae eteenpäin
\item{\Com{?\emph{haku}}} Hae taaksepäin
\item{\Com{n N}} Toista hakua eteen- / taaksepäin
\item{\Com{:s/\emph{haku}/\emph{korvaus}/}} etsi-korvaa ensimmäinen riviltä
\item{\Com{:\%s/\ldots}} etsi-korvaa ensimmäinen koko tiedostosta
\item{\Com{s/\dots/\dots/g}} etsi-korvaa kaikki esiintymät
\item{\Com{s/\dots/\dots/i}} älä välitä kirjainkoosta
\item{\Com{.}} (Haussa) jokeri
\item{\Com{.+}} (Haussa) mitä tahansa merkkejä yksi tai enemmän
\end{description}

\subsection*{Täydennykset}
Nämä komennot toimivat vain lisäysmoodissa.
\begin{description}
\item{\Com{C-n}} Etsi sanaa edestäpäin / selaa eteenpäin
\item{\Com{C-p}} Etsi sanaa takaapäin  / selaa taaksepäin
\item{\Com{C-x C-l}} Täydennä rivejä
%\item{\Com{C-x C-f}} Täydennä tiedostonimiä
%\item{\Com{C-x C-d}} Täydennä sanakirjasta
\end{description}

\subsection*{Puskurit}
Ohessa vakiokomentojen lisäksi demotut pikanäppäimet.
\begin{description}
\item{\Com{:e}} Avaa tiedosto 
\item{\com{:bn} tai \Com{C-j}} Seuraava puskuri
\item{\com{:bp} tai \Com{C-k}} Edellinen puskuri
\item{\Com{:bd}} Sulje puskuri
\item{\Com{:w}} Tallenna puskuri 
\item{\Com{:buffers}} Lista aukiolevista puskureista
\end{description}

\subsection*{Ikkunat}
Ikkunoista on moneksi. Ne ovat siis näkymiä puskureihin.
\begin{description}
\item{\Com{:sp}} Halkaise kahdeksi ikkunaksi
\item{\Com{C-w C-w}} Hyppää seuraavaan ikkunaan
\item{\Com{C-w j k h l}} Alempi/ylempi / vasen/oikea ikkuna
\item{\Com{C-w J K H L}} Siirrä ikkuna
\item{\Com{C-w + - < >}} Muuta kokoa
\item{\Com{C-w \_ |}} Maksimoi korkeus/leveys
\item{\Com{C-w =}} Jaa tasaisesti
\item{\Com{:q}} Sulje ikkuna
\end{description}

\end{multicols}
\end{document}
